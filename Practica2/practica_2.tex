\part{Práctica 2}
\section{Actividad 2-1}
\label{p21}
\begin{center}
    \parbox{12cm}{\justify\textit{Para esta práctica, utilice el conjutno de datos de altura de ola proporcionado en Moodle.\\
    Describa las operaciones de preprocesamiento que ha realizado sobre la base de datos proporcionada y cómo queda la base de datos final ya preprocesada. Se deja a su elección el conjunto de técnicas a aplicar, así como el nivel de detalle y descripción que quiera dar a su trabajo. \\   
    Para probar rendimientos sobre su preprocesamiento, puede lanzar cualquier algoritmo de Weka, por ejemplo classifiers.functions.Logistic, y fijarse en la métrica ``Correctly Classified Instances''. En la opción ``Suplied test set'' se indicaría el fichero del conjunto de test, mientras que el de entrenamiento corresponde al que se ha cargado desde la pestaña Preprocess.}}
\end{center}

\subsection{Introducción}
\subsection{Datos perdidos}
\subsection{Unificación de medidas}
\subsection{Datos extremos}
\subsection{Selección de características}
\subsection{Normalización}