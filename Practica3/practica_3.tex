\part{Práctica 3}
\section{Actividad 3-1}
\label{p31}
\begin{center}
    \parbox{12cm}{\justify\textit{Escoja una de las bases de datos de clasificación para el trabajo de las dispuestas en Moodle (Breast Cancer, Dermatology, Fantasmas, Glass, Vehicle, Wine, Zoo). \\
    Se entiende que además de pasarla a formato .arff ya ha aplicado el preprocesamiento necesario en función del fichero ``\textbf{Pistas sobre los datasets con posible preprocesamiento a simple vista.pdf}'', en el caso de que sea una de las bases de datos que lo requiera.
    \begin{itemize}
        \item Aplique el preprocesamiento adicional (si se puede aplicar sobre: 1) reemplazamiento de datos perdidos, 2) normalización y 3) paso de nominal a binario u ordinar a numérico. Explique el preprocesamiento que haya llevado a cabo en los aspectos citados, y de no tener que hacerlo explique también por qué.
    \end{itemize}}}
\end{center}

%-------------------------------------------------------------------------------
%-------------------------------------------------------------------------------
%-------------------------------------------------------------------------------
\clearpage
\section{Actividad 3-2}
\label{p32}
\begin{center}
    \parbox{12cm}{\justify\textit{Con la base de datos escogida anteriormente, use el algoritmo de clasificación \textbf{KNN} (IBK en Weka) con un 10-fold crossvalidation. Use un valor de vecinos $k=3$ dejando por defecto el resto de parámetros.
    \begin{itemize}
        \item Interprete la salida en cuanto a los valores de las métricas que proporciona Weka.\\
        Tenga en cuenta si se clasifican bien todas las clases de su problema (TP Rate por clase) y fíjese también en la matriz de confusión. \\
        Para explicar los resultados, haga uso de tablas donde se muestren los valores que está interpretando.
    \end{itemize}}}
\end{center}

%-------------------------------------------------------------------------------
%-------------------------------------------------------------------------------
%-------------------------------------------------------------------------------
\clearpage
\section{Actividad 3-3}
\label{p33}
\begin{center}
    \parbox{12cm}{\justify\textit{Con la base de datos escogida anteriormente, ejecute el algoritmo Simple-Logistic con 10-fold crossvalidation.
    \begin{itemize}
        \item Analice los modelos obtenidos, las variables que podrían ser más influyentes (valores $\beta$) variables que no se usan y métricas. \\
        Use tablas para explicar los resultados de manera que haya una lectura legible.
    \end{itemize}}}
\end{center}