\documentclass[10pt, titlepage]{article}
% preámbulo
\usepackage{lmodern}
\usepackage[T1]{fontenc}
\usepackage[document]{ragged2e}
\usepackage[spanish,activeacute]{babel}
\usepackage{mathtools}
\usepackage{amsmath}
\usepackage{hyperref}
\usepackage[table,xcdraw]{xcolor}
\usepackage{geometry}
\usepackage{float}
\usepackage{graphicx}
\usepackage{lscape} % Para intercalar páginas en horizontal

% Soporte para notas al pie referenciadas desde dentro de una tabla
\usepackage{footnote}
\makesavenoteenv{table}
\makesavenoteenv{tabular}

% Interlineado 1,5
\usepackage{setspace}
\onehalfspace


\geometry{
    a4paper,
    left=25mm,
    right=25mm,
    top=25mm,
    bottom=25mm
}
\usepackage{fancyhdr}

\setlength{\headheight}{13pt}% Quitar warning de altura de encabezado
\pagestyle{fancy}
\fancyhf{}
\lhead{\leftmark}
\rhead{\rightmark}
%\rhead{Memoria de Prácticas IAA}
\lfoot{Eduardo Arroyo Ramírez}
\rfoot{Página \thepage}

\def\code#1{\texttt{#1}}

\hypersetup{
    colorlinks=true,
    linkcolor=blue,
    filecolor=magenta,
    urlcolor=cyan,
}

\title{Introducción al Aprendizaje Automático - Memoria de prácticas}
\author{Eduardo Arroyo Ramírez}

\begin{document}

\graphicspath{ {Practica1/capturas/},{Practica2/capturas/},{Practica3/capturas/},{Practica4/capturas/} }

\makeatletter
\begin{titlepage}
    \begin{center}
        {\scshape\LARGE Escuela Politécnica Superior \par}
        \vspace{0.5cm}
        {\scshape\Large Universidad de Córdoba \par}
        
        \vspace{1.5cm}
        \includegraphics[scale=1]{machine-learning}
        \vspace{1.5cm}

        {\scshape\Huge Introducción al Aprendizaje Automático \par}
        \vspace{0.5cm}
        {\itshape\LARGE Memoria de prácticas \par}
    \end{center}
    \vspace{5cm}
    \begin{center}
        {\scshape\Large Base de datos: Fantasmas \par}
    \end{center}
    \vspace{4cm}
    \begin{flushright}
        \@author\space \\
        i12arrae@uco.es \\
        Curso 2019/2020
    \end{flushright}
\end{titlepage}

\tableofcontents
\clearpage
\listoffigures
\clearpage
\listoftables

% Aquí comienza el cuerpo del documento
\justify
%\setlength{\parindent}{1.5cm}
\part{Práctica 1}
\section{Práctica 1-1}\label{p11}
\subsection{Actividad 1}\label{p11a1}
\subsubsection{Enunciado}
El enunciado de la actividad indica lo siguiente:
\begin{center}
    \parbox{12cm}{\justify\textit{
        Elija 3 bases de datos de la UCI Machine Learning Repository de las que hay en Moodle y transformelas a .arff, indicando en cada una de ellas qué procedimiento ha seguido.
    }}
\end{center}

\subsubsection{Consideraciones generales}
Para la realización de la tarea he tomado la decisión de editar manualmente los archivos utilizando Visual Studio Code y guardarlos con la extensión \code{.arff}. El paso a csv utilizando excel que se ha sugerido en clase supone varios cambios de formato en los que pueden aparecer diversos problemas como conflictos entre el separador de columnas CSV y el separador de decimales o miles, problemas con el carácter de salto de línea, codificación, etc.

Para conocer los identificadores y tipos de los atributos de cada base de datos he consultado el archivo \code{.names} del directorio de descarga de cada base de datos, que contiene el listado de campos con su nombre, su tipo y sus posibles valores, si procede. He consultado la especificación del formato \code{.arff} en la \href{https://www.cs.waikato.ac.nz/ml/weka/arff.html}{página correspondiente del manual de weka}. Como puede comprobarse en la figura \ref{fig:ejemplo-arff}, un archivo \code{.arff} consta de tres secciones:
\begin{figure}[H]
    \caption{Ejemplo de archivo \code{.arff}}
    \centering
    \includegraphics[scale=0.65]{ejemplo-archivo-arff}
    \label{fig:ejemplo-arff}
\end{figure}
\begin{enumerate}
    \item Identificación de la base de datos. Se trata de una línea con el token \code{@relation} seguido por un espacio y el nombre de la base de datos. Por ejemplo: \code{@relation breast-cancer}.
    \item Identificación de los atributos. Tantas líneas como atributos tenga la base de datos, cada una comienza con el token \code{@atribute} seguido del nombre del atributo y el tipo. Los tipos pueden ser:
        \begin{itemize}
            \item Numéricos: \code{@attribute <nombre\string_atributo>\space numeric}
            \item Cadenas de texto: \code{@attribute <nombre\string_atributo> \space string}
            \item Listas de etiquetas: \code{@attribute <nombre\string_atributo> \space\string{valor\string_1, valor\string_2, \dots\string}}
            \item Fechas: \code{@attribute <nombre\string_atributo>\space date [formato\string_de\string_fecha]}. El formato de fecha es opcional, y por defecto acepta ISO-8601 y ``yyyy-MM-dd'T'HH:mm:ss''.
        \end{itemize}
    \item Bloque de datos. Esta sección se inicia con una línea que contiene únicamente palabra clave \code{@data}. A continuación se encontrarán los registros dispuestos en líneas y con sus atributos separados por comas, en el mismo orden en que se han especificado en la cabecera:

    \begin{center}
        \parbox{5.1cm}{\code{@data \\
            v1a1, v1a2, \dots, v1aN \\
            v2a1, v2a2, \dots, v2aN \\
            \dots \\
            vMa1, vMa2, \dots, vMaN
        }}
    \end{center}

\end{enumerate}

\subsubsection{Base de datos \code{breast-cancer}}
Tras aplicar el tratamiento mencionado en el apartado 1.1.2 al archivo \code{breast-cancer.data} se han obtenido los resultados siguientes:

\begin{enumerate}
    \item La base de datos tiene 10 atributos, de los cuales el primero es la clase, que toma los valores ``no-recurrence-events'' y ``recurrence-events''. Convendría colocar la clase al final, que es su lugar por defecto.
    \item Los atributos son nominales basados en etiquetas (por ejemplo breast\_quad, menopause\dots) o en rangos numéricos (inv\_nodes, tumor\_size\dots). El atributo deg\_malig se podría poner como numérico ya que parece representar el grado de malignidad en un rango de 1 a 3, por lo que hay una distancia distinta entre los elementos (por ejemplo 1-2 y 1-3).
\end{enumerate}
\begin{figure}[h]
    \caption{Captura del archivo \code{breast-cancer.arff}.}
    \centering
    \includegraphics[scale=0.45]{breast-cancer-arff}
\end{figure}

\begin{figure}[h]
    \caption{Archivo \code{breast-cancer.arff} cargado en Weka.}
    \centering
    \includegraphics[scale=0.35]{breast-cancer-weka}
\end{figure}

\subsubsection{Base de datos \code{dermatology}}
Tras aplicar el tratamiento mencionado en el apartado 1.1.2 al archivo 
\code{dermatology.data} se han obtenido los resultados siguientes:

\begin{enumerate}
    \item La base de datos tiene 34 atributos independientes y una clase. En total hay 366 patrones.
    \item La mayoría los atributos son de tipo numérico con valores [0-3]. En la descripción se indica que los valores indican un grado obtenido de un análisis. El caso del atributo family-history es una excepción, ya que es nominal con valores 0 y 1 y representa si alguna de las enfermedades ha sido observada en la familia. Otra excepción es la edad, que siendo numérica, no está restringida al rango anterior.
    \item La clase toma valores de 1 a 6 y cada valor representa un diagnóstico diferente, por lo que es un dato nominal.
\end{enumerate}

\begin{figure}[h]
    \caption{Captura del archivo \code{dermatology.arff}.}
    \centering
    \includegraphics[scale=0.40]{dermatology-arff}
\end{figure}

\begin{figure}[h]
    \caption{Archivo \code{dermatology.arff} cargado en Weka.}
    \centering
    \includegraphics[scale=0.35]{dermatology-weka}
\end{figure}


\subsubsection{Base de datos \code{wine}}
Tras aplicar el tratamiento mencionado en el apartado 1.1.2 al archivo 
\code{wine.data} se han obtenido los resultados siguientes:
\begin{enumerate}
\item La base de datos tiene 13 atributos independientes y una clase al principio.
\item Los atributos son numéricos continuos.
\item La clase toma valores 1, 2 o 3 con frecuencias 59, 71 y 48 respectivamente. Es un dato nominal
\end{enumerate}

\begin{figure}[h]
    \caption{Captura del archivo \code{wine.arff}.}
    \centering
    \includegraphics[scale=0.50]{wine-arff}
\end{figure}

\begin{figure}[h]
    \caption{Archivo \code{wine.arff} cargado en Weka.}
    \centering
    \includegraphics[scale=0.35]{wine-weka}
\end{figure}

\clearpage

\section{Práctica 1-2}\label{p12}
\subsection{Actividad 1}\label{p12a1}
\subsubsection{Enunciado}
El enunciado de la actividad indica lo siguiente:
\begin{center}
    \parbox{12cm}{\justify\textit{Elija 3 filtros No Supervisados de los que aparecen listados, expliquelos y describa cómo quedan los datos antes y después al aplicarlos sobre una o varias bases de datos.
    \begin{itemize}
        \item Consulte el UCI Machine Learning Repository para una descripción de la base de datos y la transformación a \code{.arff}
        \item Si no puede aplicar un filtro elegido en ninguna base de datos describa por qué, y constrúyase una base de datos ficticia y pequeña donde si pueda aplicarlo.
        \item Use capturas de pantalla, salidas de Weka y todo lo que considere necesario para sus ejercicios.
        \item La puntuación variará en función de la argumentación y dificultad de los filtros elegidos.
        \begin{enumerate}
            \item filters/unsupervised/attribute/Normalize
            \item filters/unsupervised/attribute/ReplaceMissingValues
            \item filters/unsupervised/attributes/NominalToBinary
            \item filters/unsupervised/intance/RemoveDuplicates
            \item filters/unsupervised/instance/Resample
            \item filters/unsupervised/attribute/Remove
            \item filters/unsupervised/attributes/RemoveUseless
        \end{enumerate}
    \end{itemize}
    }}
\end{center}

%-------------------------------------------------------------------------------
%-------------------------------------------------------------------------------
%-------------------------------------------------------------------------------

\subsection{Actividad 2}\label{p12a2}
\subsubsection{Enunciado}
El enunciado de la actividad indica lo siguiente:
\begin{center}
    \parbox{12cm}{\justify\textit{Elija 3 filtros Supervisados de los que aparecen listados, expliquelos y describa cómo quedan los datos antes y después al aplicarlos sobre una o varias bases de datos.
    \begin{itemize}
        \item Consulte el UCI Machine Learning Repository para una descripción de la base de datos y la transformación a \code{.arff}
        \item Si no puede aplicar un filtro elegido en ninguna base de datos describa por qué, y constrúyase una base de datos ficticia y pequeña donde si pueda aplicarlo.
        \item Use capturas de pantalla, salidas de Weka y todo lo que considere necesario para sus ejercicios.
        \item La puntuación variará en función de la argumentación y dificultad de los filtros elegidos.
        \begin{enumerate}
            \item filters/supervised/attribute/Discretize
            \item filters/supervised/attribute/NominalToBinary
            \item filters/supervised/instance/SpreadSubsample
            \item filters/supervised/instance/ClassBalancer
            \item filters/supervised/instance/Resample
        \end{enumerate}
    \end{itemize}
    }}
\end{center}
\part{Práctica 2}
\section{Práctica 2-1}\label{p21}
\subsection{Actividad 1}\label{p21a1}
\part{Práctica 3}
\section{Actividad 3-1}
\label{p31}
\begin{center}
    \parbox{12cm}{\justify\textit{Escoja una de las bases de datos de clasificación para el trabajo de las dispuestas en Moodle (Breast Cancer, Dermatology, Fantasmas, Glass, Vehicle, Wine, Zoo). \\
    Se entiende que además de pasarla a formato .arff ya ha aplicado el preprocesamiento necesario en función del fichero ``\textbf{Pistas sobre los datasets con posible preprocesamiento a simple vista.pdf}'', en el caso de que sea una de las bases de datos que lo requiera.\\
    Aplique el preprocesamiento adicional (si se puede aplicar sobre: 1) reemplazamiento de datos perdidos, 2) normalización y 3) paso de nominal a binario u ordinar a numérico. \\
    Explique el preprocesamiento que haya llevado a cabo en los aspectos citados, y de no tener que hacerlo explique también por qué.}}
\end{center}

%-------------------------------------------------------------------------------
%-------------------------------------------------------------------------------
%-------------------------------------------------------------------------------
\clearpage
\section{Actividad 3-2}
\label{p32}
\begin{center}
    \parbox{12cm}{\justify\textit{Con la base de datos escogida anteriormente, use el algoritmo de clasificación \textbf{KNN} (IBK en Weka) con un 10-fold crossvalidation. Use un valor de vecinos $k=3$ dejando por defecto el resto de parámetros.\\
    Interprete la salida en cuanto a los valores de las métricas que proporciona Weka.\\
    Tenga en cuenta si se clasifican bien todas las clases de su problema (TP Rate por clase) y fíjese también en la matriz de confusión. \\
    Para explicar los resultados, haga uso de tablas donde se muestren los valores que está interpretando.}}
\end{center}

%-------------------------------------------------------------------------------
%-------------------------------------------------------------------------------
%-------------------------------------------------------------------------------
\clearpage
\section{Actividad 3-3}
\label{p33}
\begin{center}
    \parbox{12cm}{\justify\textit{Con la base de datos escogida anteriormente, ejecute el algoritmo Simple-Logistic con 10-fold crossvalidation. \\
    Analice los modelos obtenidos, las variables que podrían ser más influyentes (valores $\beta$) variables que no se usan y métricas. \\
    Use tablas para explicar los resultados de manera que haya una lectura legible.}}
\end{center}
\part{Práctica 4}
\section{Actividad 4-1}
\label{p41}
\begin{center}
    \parbox{12cm}{\justify\textit{Escoja una de las bases de datos de clasificación para el trabajo de las dispuestas en Moodle (Breast Cancer, Dermatology, Fantasmas, Glass, Vehicle, Wine, Zoo). \\
    Se entiende que además de pasarla a formato .arff ya ha aplicado el preprocesamiento necesario en función del fichero ``\textbf{Pistas sobre los datasets con posible preprocesamiento a simple vista.pdf}'', en el caso de que sea una de las bases de datos que lo requiera.
    \begin{itemize}
        \item Cargue la base de datos y ejecute el algoritmo \textbf{C4.5} usando un 75\% para entrenar y un 25\% para generalizar, con los parámetros por defecto.
        \item Analice y muestre el árbol obtenido con los parámetros por defecto: nodo principal, número de nodos u hojas, variables presentes y omitidas. Comente también los resultados de las métricas obtenidas.
    \end{itemize}
    }}
\end{center}


\clearpage
\section{Actividad 4-2}
\label{p42}
\begin{center}
    \parbox{12cm}{\justify\textit{Escoja una de las bases de datos de clasificación para el trabajo de las dispuestas en Moodle (Breast Cancer, Dermatology, Fantasmas, Glass, Vehicle, Wine, Zoo). \\
    Se entiende que además de pasarla a formato .arff ya ha aplicado el preprocesamiento necesario en función del fichero ``\textbf{Pistas sobre los datasets con posible preprocesamiento a simple vista.pdf}'', en el caso de que sea una de las bases de datos que lo requiera.
    \begin{itemize}
        \item Cargue la base de datos con un 75/25\% y ejecute el algoritmo Multilayer-Perceptron con los valores por defecto.
        \item ¿Qué observa al ir modificando sólo el \textbf{TrainingTime}? ¿Cambia el valor de Correctly Classified instances al modificar el parámetro? ¿Se estanca el aprendizaje o sobreentrena?
        \item ¿Qué observa al ir modificando sólo el \textbf{LearningRate}? ¿Cambia el valor de Correctly Classified instances al modificar el parámetro? ¿Se estanca el aprendizaje o sobreentrena?
\end{itemize}
    }}
\end{center}
\end{document}
